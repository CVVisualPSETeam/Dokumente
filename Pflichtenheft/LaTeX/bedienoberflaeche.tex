\begin{itemize}
	\item Die Bedienoberfl�che wird in Qt implementiert sein.
	\item Sie wird aus einem Hauptfenster pro Thread bestehen.
	\item Die verschiedenen Debug-Ansichten werden entweder im Hauptfenster oder in einzelnen Fenstern dargestellt.
	\item Daten von vorherigen API-Aufrufen werden auf einer �bersichtsseite tabellarisch dargestellt
\end{itemize}


\subsection{Skizzen}
\label{sketches}

\begin{sketch}[�bersichtseite mit Tabs, ohne Vorschaubilder]{overview_tabs_wo_images}
Der Benutzer kann die einzelnen Datens�tze ausw�hlen und mit Hilfe der Suchleiste in der Tabelle zu suchen.
Wenn er auf einen Datensatz in der Tabelle dr�ckt, �ffnet er die Visualisierung f�r diesen Datensatz.
Au�erdem kann er mit dem Knopf \button{Show Images} Vorschaubilder in der Tabelle anzeigen lassen (n�chste Skizze) und �ber das Menu \button{View} einstellen, ob er die Visualisierungen als Tabs im Hauptfenster oder als eigene Fenster angezeigt haben m�chte.
\end{sketch}


\begin{sketch}[�bersichtseite mit Tabs und Vorschaubildern]{overview_tabs_images}
Das gleiche �bersichtsfenster, nur dass in der Tabelle Vorschaubilder angezeigt werden.
Durch �berfahren jener mit der Maus, werden gr��ere Bilddarstellungen eingeblendet.
\end{sketch}


\begin{sketch}[�bersichtseite mit Fenstern und Vorschaubildern]{overview_windows_images}
�bersichtseite mit Fenstern statt Tabs und Vorschaubildern. Dies erm�glicht dem Benutzer mehrere Visualisierungen nebeneinander anzusehen - und damit auch mehrere Bildschirme zu nutzen.
\end{sketch}


\begin{sketch}[Basisvisualisierung eines Match-Datensatzes]{test_match_window}
Unten in der Fensterleiste findet sich der aktuelle Zoomfaktor und die aktuelle Position in einem der beiden Bilder.
Wichtig: Die Position entspricht jener im gesamten Bild - nicht jener im angezeigten Ausschnitt.

Mit dem Button \button{View Raw} kann der Benutzer die Rohdatenanzeige f�r den aktuellen Datensatz anzeigen.
Der Button \button{Step further} wird angezeigt, sofern die zu debuggende Anwendung wegen dieser Visualisierung blockiert. Wobei damit
die Visualisierung nicht geschlossen wird.

Die View-Auswahl und die Zoom-Kn�pfe am rechten Rand sind selbsterkl�rend. Bei Eindr�cken des
Schlossknopfes versucht das Programm die Anzeigen beider Bilder zu synchronisieren.
\end{sketch}


\begin{sketch}[Rohdatenanzeige]{test_match_raw_window}
Tabellarische Anzeige der Rohdaten, nach dem der Benutzer in der vorherigen Skizze auf den Knopf \button{View Raw} gedr�ckt hat.

Dem Benutzer ist es wieder m�glich mit Hilfe der Suchleiste in den Daten zu suchen und mit gedr�ckter \textit{STRG}-Taste auch mehrere Eintr�ge zu markieren. �ber das Kontextmenu ist es des Weiteren m�glich entweder die ausgew�hlten Eintr�ge in der Visualisierung hervorzuheben oder sie als \textit{JSON} oder \textit{CSV} in die Zwischenablage zu kopieren.
\end{sketch}


\begin{sketch}[Hohe Zoomstufen mit Zusatzinformationen]{test_match_deep_zoom_window}
Der Benutzer sieht bei einem hohen Zoomfaktor Zusatzinformationen im Bild, in diesem Fall die Werte f�r Rot, Gr�n, Blau f�r jedes einzelne Pixel.
\end{sketch}


\begin{sketch}[Projektion]{test_match_area_tab}
Projektion einer automatischen Gruppierung von Matches von einem auf das andere Bild.
Hierbei wird jeweils das Histogramm des vom Benutzer ausgew�hlten Bereichs angezeigt.
\end{sketch}


\begin{sketch}[Darstellung von Punkttranslationen]{match_movement_parrot_window}
Darstellung von Punkttranslationen von einem Bild zum anderen mit Pfeilen.
Die L�nge und Richtung des jeweiligen Pfeils entsprechen der Translation oder Verschiebung des Startpunktes im anderen Bild.
\end{sketch}