
\begin{center}
\includegraphics[width=0.3\linewidth]{./images/OpenCV_Logo_with_text.png}
\end{center}

\vspace{2cm}

OpenCV ist eine Bilderkennungs-Bibliothek. Sie wurde im Jahre 2000 der breiteren �ffentlichkeit vorgestellt, 6 Jahre sp�ter erschien Version 1.0.  

Seitdem hat sich das Projekt stetig weiterentwickelt - sie ist heute quasi der Standard unter den freien Bilderkennungs-Bibliotheken. Sie umfasst heutzutage Algorithmen f�r zum Beispiel einfache Bilderkennung, Gesichtserkennung, Bewegungsverfolgung und vieles mehr. Die Bibliothek ist auf Performance optimiert und findet auch deshalb heute Einsatz in vielen Bereichen, wie etwa der Augmented Reality auf mobilen Ger�ten  
oder an Universit�ten.  

Wer jedoch damit arbeitet, steht zur Zeit noch vor einem Problem:  

Es existiert keine richtige Debug-Visualisierung f�r OpenCV. Das bedeutet, dass es augenblicklich keine Bibliothek (und auch kein Werkzeug) gibt, mit welchem man sich ad�quat visualisieren lassen kann, was zum Beispiel eine lange Liste von Matches mit den zwei dazugeh�rigen Bildern zu tun hat, oder welche Auswirkung ein bestimmter Filter hat. Deswegen fangen Entwickler oft an, eigene L�sungen f�r dieses Problem zu entwickeln, die in der Hauptsache aus dem einfachen Speichern der Bilder oder der Verwendung primitiver Methoden von OpenCV  (wie etwa 'imshow()', das einfach nur ein Bild anzeigt) in Kombination mit schnell zusammengeschusterten Zeichenroutinen  
bestehen. 

Besonders f�r Neulinge stellt dieses weitestgehende Fehlen von Visualisierungsm�glichkeiten zu Debugzwecken eine gro�e H�rde dar.  

\vspace{0.5cm}

Daran etwas zu �ndern ist Ziel unserer Arbeit als PSE-Team an einer Open Source-Visualisierung f�r OpenCV.