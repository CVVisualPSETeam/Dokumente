Unser Projekt ist eine reine Debug-Bibliothek, die es dem Benutzer m�glichst einfach macht, seine Bilddaten zu visualisieren. Deshalb enth�lt der Funktionsumfang im Grunde genommen nur Visualisierungen und die meisten Testszenarios besitzen auf Grund dessen die folgende Struktur:
\begin{enumerate}
	\item Der Programmierer schreibt den gew�nschten API-Aufruf an die gew�nschte Stelle in seinen Quelltext.
	\item Ein Fenster �ffnet sich und visualisiert die beim Aufruf �bergebenen Daten.
	\item Nun kann er mit der Visualisierung arbeiten, sie in den Einstellungen anpassen oder die Visualisierung wechseln.
	\item Bei blockierenden Aufrufen klickt er nun auf einen Button, der das aufrufende Programm weiterlaufen l�sst.
	\item Er sucht entweder weiter Fehler in seinem Programm, oder sieht die angefallenen Datens�tze in der �bersichtsseite durch und visualisiert die gew�nschten.
	\item Die Debugumgebung l�sst sich ohne Fehler beenden.
\end{enumerate}

\subsection{Tests f�r Filterview}
\begin{description}
	\item[/TF010/] Anzeige des Bildes/der Bilder
	\item[/TF020/] Ausw�hlen und Wechseln der Filter
	\item[/TF030/] �ndern der Parameter eines Filters
	\item[/TF040/] Bedienung bei einem gro�en Bild und geringer Rechenleistung
	\item[/TF050/] Benutzung mit mehr als einem Bild
\end{description} 

\subsection{Tests f�r Matchview}
\begin{description}
	\item[/TF060/] Anzeige des Bildes/der Bilder
	\item[/TF070/] Ausw�hlen und Wechseln von Views
	\item[/TF080/] S�mtliche Buttons und Schieberegler testen
	\item[/TF090/] Bedingung mit gro�en Bildern und mit geringer Rechenleistung
\end{description}

\subsection{Beispielhafter Ablauf}
Der Benutzer tippt irgendwo in seinem Code die folgende Zeile, wobei \texttt{inputImg1} und \texttt{inputImg2} zwei Bilder und \texttt{matches} die von OpenCV gefundenen Matches sind.
\begin{verbatim}
cvvisual::showMatches(inputImg1, inputImg2, matches);
\end{verbatim}
Dieser Code erzeugt ein Fenster, mit dem Standardview f�r Matches. Der View zeigt die Bilder nebeneinander an und zeichnet zwischen den einzelnen Matches Pfeile (siehe \nameref{test_match_window}). 
Der Benutzer hat nun die Auswahl sich zus�tzliche Informationen anzeigen zu lassen, wie den \textit{Matchwert} der einzelnen Matches durch Einf�rben der Pfeile in Falschfarben anzeigen zulassen, oder zu einem anderen View mit anderen Optionen zu wechseln.
In dem anderen View werden dieselben Bilder benutzt.
Jetzt dr�ckt er den \button{View Raw} Knopf um sich sich gesamten Rohdaten (wie beispielsweise die Punkt-Koordinaten der einzelnen Matches) im nun aufgehenden Fenster anzeigen zu lassen (siehe \nameref{test_match_raw_window}).
F�r eine angenehmere Bedingung steht ihm eine Suchleiste mit einigen Befehlen zur Verf�gung. Mit ihrer Hilfe sortiert er die Liste nach der X-Koordinate des ersten Punktes jedes Matches.
Er w�hlt die erste Zeile der Tabelle durch Rechtsklicken aus und exportiert mit dem erscheinenden Kontextmenu den Zeileninhalt - ein Match - in die Zwischenablage als JSON. Dieses JSON schreibt er f�r sp�ter in die Ursprungsdatei.
Nun kehrt der Benutzer durch Klicken des \button{Return to match view} Knopfes zur Matchview zur�ck und zoomt dort etwas im Bild herum und untersucht f�r ihn wichtige Bildteile. 
Nachdem der Benutzer, das Gew�nschte gesehen hat dr�ckt er den \button{Step further} Knopf, damit sein Programm weiter l�uft.
Wenn sein Programm sich beendet, beendet sich auch CVVisual und schlie�t alle Fenster.
