Unser Produkt wird ein Debug-Werkzeug f�r diverse OpenCV-Funktionalit�t sein. Hierzu werden wir die Ergebnisse von Filteroperationen verwenden, um die Auswirkungen des Filters auf das Ursprungsbild zu visualisieren.

Um das Debuggen zu erleichtern und Code�nderungen im Anschluss �berfl�ssig zu machen, werden wir die Funktionalit�t dabei so implementieren, dass pro Translation Unit der Debug-Modus sowohl w�hrend des Kompiliervorgangs als auch zur Laufzeit (de-)aktiviert werden kann. Bei der Deaktivierung w�hrend des Kompilierens werden wir hierbei versuchen die Programmlaufzeit in keiner Weise negativ zu beeinflussen.

Bei der Verwendung in Programmen mit mehreren Threads wird zwar mit eingeschr�nktem Komfort zu rechnen
sein, aber die prinzipielle Funktionalit�t an sich wird unbeeintr�chtigt bleiben.

Zur Verwendung werden wir f�r die debugbaren OpenCV-Funktionalit�ten Funktionen bereitstellen, welche
eine graphische Darstellung des Filters zu einem gro�en Debug-Hauptfenster hinzuf�gen werden (pro
Thread ein Hauptfenster).

Je nach View werden wir beispielsweise die Matches zwischen zwei Bildern mit Pfeilen darstellen.
Hierbei soll es auch eine M�glichkeit geben, die Darstellungen selbst zu filtern, beispielsweise indem
nur Pfeile zwischen Unterschieden, die einen gewissen Schwellwert �berschreiten, gezeichnet werden.