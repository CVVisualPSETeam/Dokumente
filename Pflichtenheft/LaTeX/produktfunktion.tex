Unser Produkt ist ein Debug-Werkzeug f�r diverse OpenCV-Funktionalit�t. Hierzu werden die Ergebnisse von Filteroperationen verwenden, um die Auswirkungen des Filters auf das Ursprungsbild zu visualisieren.

Um das Debuggen zu erleichtern und Code�nderungen im Anschluss �berfl�ssig zu machen, ist die Funktionalit�t dabei so implement, dass pro Translation Unit der Debug-Modus sowohl w�hrend des Kompiliervorgangs als auch zur Laufzeit (de-)aktiviert werden kann. Bei der Deaktivierung w�hrend des Kompilierens wird hierbei die Programmlaufzeit in keiner Weise negativ beeinflusst.

Bei der Verwendung in Programmen mit mehreren Threads ist zwar mit eingeschr�nktem Komfort zu rechnen,die prinzipielle Funktionalit�t an sich bleibt dennoch unbeeintr�chtigt.

Zur Verwendung stellen wir f�r die debugbaren OpenCV-Funktionalit�ten Funktionen bereit, welche eine graphische Darstellung des Filters zu einem gro�en Debug-Hauptfenster hinzuf�gen (pro
Thread ein Hauptfenster).

Je nach View stellen wir beispielsweise die Matches zwischen zwei Bildern mit Pfeilen dar.
Hierbei gibt es auch eine M�glichkeit, die Darstellungen selbst zu filtern, beispielsweise indem nur Pfeile zwischen Unterschieden, die einen gewissen Schwellwert �berschreiten, gezeichnet werden.