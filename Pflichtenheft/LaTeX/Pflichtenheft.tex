%% Basierend auf einer TeXnicCenter-Vorlage von Mark M�ller
%%%%%%%%%%%%%%%%%%%%%%%%%%%%%%%%%%%%%%%%%%%%%%%%%%%%%%%%%%%%%%%%%%%%%%%

% W�hlen Sie die Optionen aus, indem Sie % vor der Option entfernen  
% Dokumentation des KOMA-Script-Packets: scrguide

%%%%%%%%%%%%%%%%%%%%%%%%%%%%%%%%%%%%%%%%%%%%%%%%%%%%%%%%%%%%%%%%%%%%%%%
%% Optionen zum Layout des Artikels                                  %%
%%%%%%%%%%%%%%%%%%%%%%%%%%%%%%%%%%%%%%%%%%%%%%%%%%%%%%%%%%%%%%%%%%%%%%%
\documentclass[%
%a5paper,							% alle weiteren Papierformat einstellbar
%landscape,						% Querformat
%10pt,								% Schriftgr��e (12pt, 11pt (Standard))
%BCOR1cm,							% Bindekorrektur, bspw. 1 cm
%DIVcalc,							% f�hrt die Satzspiegelberechnung neu aus
%											  s. scrguide 2.4
%twoside,							% Doppelseiten
%twocolumn,						% zweispaltiger Satz
%halfparskip*,				% Absatzformatierung s. scrguide 3.1
%headsepline,					% Trennline zum Seitenkopf	
%footsepline,					% Trennline zum Seitenfu�
%titlepage,						% Titelei auf eigener Seite
%normalheadings,			% �berschriften etwas kleiner (smallheadings)
%idxtotoc,						% Index im Inhaltsverzeichnis
%liststotoc,					% Abb.- und Tab.verzeichnis im Inhalt
%bibtotoc,						% Literaturverzeichnis im Inhalt
%abstracton,					% �berschrift �ber der Zusammenfassung an	
%leqno,   						% Nummerierung von Gleichungen links
%fleqn,								% Ausgabe von Gleichungen linksb�ndig
%draft								% �berlangen Zeilen in Ausgabe gekennzeichnet
]
{scrartcl}

\usepackage[ngerman]{babel}
\usepackage[T1]{fontenc}
\usepackage[ansinew]{inputenc}
\usepackage{graphicx}
\usepackage{calc}
\usepackage[vmargin=3cm, hmargin=2cm]{geometry}
\usepackage{lmodern}
\usepackage{float}
\usepackage{graphicx}
\usepackage{nameref}
\usepackage{hyperref}
\usepackage{subcaption}
\usepackage{multirow}
\usepackage{booktabs}
\usepackage{ifthen}
\usepackage{enumitem}
\usepackage{geometry}
\usepackage{scalefnt}

%macro for buttons
\usepackage{tikz}
\usetikzlibrary{shadows}
\tikzstyle{buttonstyle} = [rectangle, fill = black!30, draw = black!80, drop shadow, font={\sffamily\bfseries}, text=white, inner sep = 3pt
]

\newcommand{\button}[1]{
{\scalefont{0.8}
\begin{tikzpicture}
\node[buttonstyle] {\ {#1}};
\end{tikzpicture}
}
}


%some neat macros...

% \scetchImg{title}{file name}
\newcommand{\sketchImg}[2]{
	\begin{figure}[H]
		\centering
		\includegraphics[width=\linewidth]{../GUI/#2}
		\caption{#1}
	\end{figure}
}

% \begin{scetch}{title}{img file}
\newenvironment{sketch}[2][\skip]{
	\subsubsection{#1}
	\label{#2}
	\sketchImg{#1}{#2}
}

\newcommand{\sketchref}[1]{
	\textsc{\nameref{#1}}
}

\newcommand{\glossaryItem}[2]{
	\item[#1] \label{#1} #2
}

%reference to a glossary entry
\newcommand{\gl}[1]{
	\rightarrow\textcolor{glossaryReferenceColor}{\nameref{#1}}
}

\newcommand{\glossaryItemWD}[4]{
	\glossaryItem{#1}{#2}\\
	\textsl{\rightarrow \href{#4}{#3}}
}

\setdescription{leftmargin=\parindent,labelindent=\parindent}

\makeindex

\setlength{\parskip}{0pt}

\makeatother

\hypersetup{
	colorlinks,
	citecolor = black,
	filecolor = black,
	linkcolor = black,
	urlcolor = black
}

\author{Johannes Bechberger, Erich Bretn�tz,\\
 Nikolai Ga�ner, Raphael Grimm,\\
  Clara Scherer, Florian Weber}
\title{%\vspace{1cm}
         \includegraphics[scale=0.4]{../../Logo/logo.pdf}\\
         \vspace{5mm}
         CVVisual -- Eine Visualisierung f�r OpenCV
         \vspace{5mm}}
\subtitle{Pflichtenheft}
\date{\today}

\geometry{a4paper,left=20mm,right=20mm, top=2cm, bottom=3cm}


\setcounter{tocdepth}{2} % hide subsubsections in table of contents

\begin{document}
\clearpage\maketitle
\thispagestyle{empty}

\pagebreak

\input einleitung
\thispagestyle{empty}

\pagebreak

\pagestyle{headings}

\tableofcontents
\thispagestyle{empty}


\pagebreak

%\section{Einleitung}
%\input einleitung

\pagebreak

\section{Produktfunktionen}
\input produktfunktion

\section{Produkteinsatz}
Die Software soll zun�chst im universit�ren Forschungsumfeld des beauftragenden Institutes eingesetzt werden.
Sp�ter kann der Nutzerkreis potenziell auf alle OpenCV Benutzer ausgedehnt werden, welche OpenCV mit Qt Unterst�tzung kompiliert haben.

\section{Produktumgebung}
Nach M�glichkeit alle Plattformen auf denen moderne Versionen von OpenCV und Qt5 laufen, sowie ein
C++11-Compiler existiert.

\pagebreak

\section{Funktionale Anforderungen}
\input funktionale_anforderungen

\section{Nichtfunktionale Anforderungen}
\input nichtfunktionale_anforderungen

\section{Produktdaten}
\label{produktdaten}
\input produktdaten

\pagebreak

\section{Systemmodell}
\input systemmodell

\pagebreak

\section{Bedienoberfl�che}
\input bedienoberflaeche

\pagebreak

\section{Testf�lle und Testszenarien}
\input tests

\section{Abgrenzungskriterien}
Unser Projekt grenzt sich durch existentes Design gegen�ber "`random code"' ab. Dar�ber hinaus ist uns
keine andere OpenSource-L�sung bekannt, die mit unserer vergleichbare Ziele verfolgt.

\textbf{Wichtig}: Unser Projekt ist kein Stand-Alone-Programm und wird voraussichtlich keinen r�ckl�ufigen Datenfluss unterst�tzen.

\section{Entwicklungsumgebung}
\input entwicklungsumgebung

\section{Lizenz}
\input lizenz

\pagebreak

\section{Glossar}
\label{sec:glossar}
\input glossar

\section{Literatur}
\input literatur

\end{document}
